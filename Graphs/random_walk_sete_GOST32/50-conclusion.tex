\Conclusion

Результаты вычислительных экспериментов и теоретических расчетов для графа демонстрируют высокую степень согласованности, что свидетельствует о правильности и эффективности применяемых моделей и алгоритмов. Тестирование проводилось на графе с четырьмя вершинами, на котором было установлено, что теоретические предсказания совпадают с практическими результатами. Кроме того, применение данных методов на графе Московского метро показало, что алгоритмы могут эффективно работать не только в теоретических условиях, но и с реальной транспортной сетью. Среднее время первого попадания, прохождения и обхода всего графа, а также эффективное сопротивление, полученные в результате симуляций, находятся в хорошем согласии с теоретическими расчетами. Эти два случая подтверждают, что предложенные модели и алгоритмы могут быть успешно применены для анализа и прогнозирования поведения более сложных и масштабных транспортных сетей.

Высокая степень совпадения теоретических и практических результатов свидетельствует о том, что даже при увеличении размера графа выбранные алгоритмы остаются корректными и эффективными. Эти результаты позволяют сделать вывод о возможности масштабирования используемых методов на более крупные графы, сохраняя их точность и надежность.
