\Introduction

Цель данной работы – это изучить применения теории графов для моделирования и анализа электрических цепей, представленных в виде графов, а также определение среднего времени перемещения между двумя вершинами графа с использованием метода случайного блуждания. Мы рассматриваем однородное случайное блуждание, при котором каждая вершина выбирает одну из своих соседних вершин с равной вероятностью на каждом шаге.

Электрические сети могут быть эффективно представлены с помощью графов, где вершины соответствуют узлам сети, а ребра — проводникам или линиям связи между этими узлами. Одним из ключевых аспектов анализа таких сетей является расчет сопротивления между двумя узлами, что имеет важное значение для оценки характеристик сети и её надежности. Этот расчет можно провести с использованием законов Кирхгофа, но альтернативный подход предполагает использование методов теории графов и случайного блуждания.

В данной работе мы реализуем метод моделирования случайного блуждания по графу для вычисления среднего времени посещения указанной вершины и среднего времени обхода всего графа. Этот метод основывается на предположении, что время перемещения по каждому ребру равно единице, а выбор следующей вершины происходит случайно с равной вероятностью для всех соседей текущей вершины. Интересным является тот факт, что среднее время перемещения тесно связано с электрическим сопротивлением между узлами графа, если сопротивление каждого ребра также принять равным единице.

В отчете была разработана программа на языке программирования Octave, которая имитирует случайные блуждания.